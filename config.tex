% Config
\newcommand*{\group}{Gruppe}
\newcommand*{\module}{Fach}
\newcommand*{\image}{}
\newcommand*{\students}{
	(Max Mustermann, 123 456)
}

% Workspace settings
% LTeX: enabled=false

\documentclass[
	a4paper,
	12pt,
]{article}

\usepackage{scrhack} % addition to koma script; disables warning concerning  \float@addtolists  in package listings

% Language settings
\usepackage[ngerman]{babel}
\selectlanguage{ngerman}

% Package for text and input type
\usepackage[T1]{fontenc}
\usepackage[utf8]{inputenc} %Linux: [utf8]; MAC: [macce]; Windows: [utf8]

\setlength{\unitlength}{1mm} %one unit is equal 1mm
\tolerance = 9000 % avoid that text gets wider the textwith
\setlength\parindent{0pt} %no indentation of paragraphs

\usepackage{
	graphicx, % include graphics
	listings, % for listings
	enumerate,% for enumerations
	epstopdf, % convert eps figures to pdf
	multirow,  % for multirows and columns in tables
	longtable % for tables longer than one page
}

% better floats for floating around figures and tables
\usepackage{float}
\restylefloat{table}
\restylefloat{figure}

% define how captions should look like
\usepackage[bf,small]{caption}

\usepackage{fancyhdr}
\pagestyle{fancy}
\fancyhf{}
\rfoot{Seite \thepage}
\setlength{\headheight}{15pt}

% avoid errors due to old font commands in gerapali
\DeclareOldFontCommand{\sc}{\normalfont\scshape}{\@nomath\sc}
\DeclareOldFontCommand{\em}{\normalfont\itshape}{\mathit}

% Extract branch name from git HEAD 
% from https://tex.stackexchange.com/questions/455396/how-to-include-the-current-git-commit-id-and-branch-in-my-document
\usepackage{xstring, catchfile}
\CatchFileDef{\headfull}{.git/HEAD.}{}
\StrGobbleRight{\headfull}{1}[\head]
\StrBehind[2]{\head}{/}[\branch]

% Loop students for titlepage
\usepackage{xinttools}

% Document settings
\def\authors{}
\xintForpair #1#2 in \students \do {\edef\authors{\authors#1 #2, }}

\usepackage{xstring}
\StrGobbleRight{\authors}{2}[\authors] % remove last two characters of generated string

\usepackage[
	pdftitle={Hausaufgabe \branch},
	pdfauthor={\authors},
	pdfpagemode=UseOutlines,
	colorlinks=true,
	linkcolor=black,
	urlcolor=black,
	citecolor=black,
	plainpages=false,
	hypertexnames=false,
	pdfpagelabels=true,
	hyperfootnotes=false,
	hyperindex=true
	]{hyperref}
	
% Important packages to use
\usepackage{pdfpages, graphicx, multirow, listings, enumerate, amsmath, amssymb, amsfonts, tikz}

% Helper functions
\newcommand*{\notundefined}[2]{
	{\IfStrEq{#1}{}{}{#2}}
}

% Check document mode
\newtoggle{includeAssignments}

\usepackage{xstring}
\IfStrEq{\jobname}{\detokenize{hausaufgabe}}{\toggletrue{includeAssignments}}{\togglefalse{includeAssignments}}

% Document image
\usepackage{tikzpagenodes}

% PDF extractor
% https://tex.stackexchange.com/questions/7653/how-to-iterate-through-the-name-of-files-in-a-folder
\usepackage{texosquery, xinttools}

\iftoggle{includeAssignments}
{
	\TeXOSQueryFilterFileList{\result}{,}{.+\string\.pdf}{./aufgaben}
}

% Comment function for hausaufgabe.pdf
\newcommand*{\comment}[1]{
	\iftoggle{includeAssignments}{
		\begin{tabular}{|p{10cm}}
			#1
		\end{tabular}
	}
}

% Begin document with titlepage
% LTeX: enabled=true
	
\begin{document}

\begin{titlepage}
	\begin{center}
		\begin{Large}
			Tutorium: \group \\
			\textbf{\module-Übungsblatt:} \branch \\

			\vspace{2.5cm}

			\begin{tabular}{r l}
				\xintForpair #1#2 in \students \do
				{\bfseries #1, & #2 \\}
			\end{tabular}

			\vspace{3cm}

			\notundefined{\image}{
				% Display image if it exists
				\begin{tikzpicture}[remember picture, overlay, shift={(current page.south)}]
					\node[anchor=south, yshift=10cm]{
						\includegraphics[
							width=5cm,
							height=5cm,
							keepaspectratio,
						]{\image}
					};
				\end{tikzpicture}
			}

		\end{Large}
	\end{center}

	\vspace*{\fill}

	\begin{flushright}
		Aachen, \today
	\end{flushright}
\end{titlepage}

\iftoggle{includeAssignments}
{
	% Load assignments
	\notundefined{\result}{
		\xintFor #1 in \result \do
		{\includepdf[pages=-]{./aufgaben/#1}}
	}
}

\lhead{Aufgabe 1}
\rhead{}

\subsection*{a)}
Sachen zur Aufgabe 1 A...

\subsection*{b)}
\subsubsection*{Nummerierte Auflistung}
\begin{enumerate}
	\item Eine einfache Auflistung
	\item Code im Fließtext mit \texttt{x = 3}
	\item \begin{enumerate}
		      \item Eine verschachtelte Auflistung
	      \end{enumerate}
\end{enumerate}

\newpage

\lhead{Aufgabe 2}
\rhead{A}

Sachen zur Aufgabe 2 A...

\subsubsection*{Mit Buchstaben}
\begin{enumerate}
	\item A
	\item B
	\item C
\end{enumerate}

\subsubsection*{Mit Aufzählungspunkten}
\begin{itemize}
	\item Punkt 1
	\item Punkt 2
\end{itemize}


\end{document}
